\section{Proper Methodology}
The problem is formulated as a normal form game where the players are the two hospitals. 
Each hospital is given \( N_A \) and \( N_B \) number of strategies where \( N_A \) 
and \( N_B \) are the total capacities of the hospitals. 
In other words, depending on the capacity of each hospital, they may choose to stop 
receiving patients from arriving ambulances whenever they reach a certain capacity 
threshold. 
The goal of this problem is to satisfy the ED regulations which state that 95\% 
of the patients should see a specialist within 4 hours of their arrival to the hospital. 
The mean of the random variable \( W_q \) is the average waiting time in the queue 
for hospital i.


\begin{equation}
     W_q(\lambda_i, \mu_i, \hat{c_i}) = \frac{1}{\hat{c_i} \mu_i} 
     \frac{(\hat{c_i} \rho_i) ^ {\hat{c_i}}}{\hat{c_i}! (1 - \rho_i) ^ 2}P_0, 
     \quad i \in \{A,B\}
\end{equation}

Thus, the utilities of the two players should be the proportion of people that fall 
within the 4 hours target. 
This is also equivalent to the probability of the waiting time of an individual 
to be less than or equal to 4 hours. 

\begin{equation}
    P(W_q(\lambda_i, \mu_i, \hat{c_i}) \leq 4), \quad i \in \{A,B\}
\end{equation}

Therefore, a sensible goal for each player should be to minimise that probability, 
but the actual target of the hospitals is to satisfy 95\% of those patients within 
the 4-hour time limit. Therefore, the goal should be to get that probability as 
close to 0.95 as possible. 
Thus each player should aim to minimise:

\begin{equation}
    |0.95 - P(W_q(\lambda_i, \mu_i, \hat{c_i}) \leq 4)|, \quad i \in \{A,B\}
\end{equation}

The classic formulation of a normal form game looks into the maximisation of each 
player's payoff. 
Consequently the utilities can be altered such that the goal of each player is to 
maximise:

\begin{align}\label{Utilities}
    U_{\hat{c_A}, \hat{c_B}} ^ {A} = 1 - |0.95 - P(W_q(\lambda_A, \mu_A, \hat{c_A}) \leq 4)| \\
    U_{\hat{c_A}, \hat{c_B}} ^ {B} = 1 - |0.95 - P(W_q(\lambda_B, \mu_B, \hat{c_B}) \leq 4)|
\end{align}

Finally, the problem can be expressed as a normal form game with two players where 
each player/hospital has \( N_A \) and \( N_B \) strategies respectively. 
The two \( N_A \times N_B \) payoff matrices for the utilities of the two hospitals 
can be defined as:

\begin{table}[h]
    \centering
    \begin{minipage}{.5\linewidth}
        A = 
        \begin{tabular}{|l|l|l|l|}
            \hline
            \( U_{1,1}^A \) & \( U_{1,2}^A \) & \dots & \( U_{1,C_2}^A \) \\ \hline
            \( U_{2,1}^A \) & \( U_{2,2}^A \) & \dots & \( U_{2,C_2}^A \) \\ \hline
            \vdots & \vdots & \( \ddots \) & \vdots \\ \hline
            \( U_{C_1,1}^A \) & \( U_{C_1,2}^A \) & \dots & \( U_{C_1,C_2}^A \) \\ \hline
        \end{tabular}
    \end{minipage}%
    \begin{minipage}{.5\linewidth}
        B = 
        \begin{tabular}{|l|l|l|l|}
            \hline
            \( U_{1,1}^B \) & \( U_{1,2}^B \) & \dots & \( U_{1,C_2}^B \) \\ \hline
            \( U_{2,1}^B \) & \( U_{2,2}^B \) & \dots & \( U_{2,C_2}^B \) \\ \hline
            \vdots & \vdots & \( \ddots \) & \vdots \\ \hline
            \( U_{C_1,1}^B \) & \( U_{C_1,2}^B \) & \dots & \( U_{C_1,C_2}^B \) \\ \hline
        \end{tabular}
    \end{minipage}
\end{table}  
Once the certain strategies of the game have been selected the ambulance service 
can decide what would be the optimal way to distribute patients. 
However, the way the ambulance service distributes patients can affect the utilities 
of the game. So how would one solve this kind of problem?
 
\subsection{Solution}
As mentioned before the problem requires the construction of two queuing models 
that will be needed for the formulation of the normal form game. 
Based on those utilities the ambulance service will then decide the percentage of 
patients that will distribute to each hospital. 

First and foremost, the model consists of several parameters that are unknown and 
are assumed to be fixed. 
The model will be run multiple times for various values of these parameters.


\begin{table}[h]
    \centering
    \begin{tabular}{|l|l|}
        \hline
        \( \Lambda \) & Number of patients that need to be distributed \\ \hline
        \( \lambda_i^o \) & Arrival rate of other patients that enter hospital i \\ \hline
        \( \mu_i \) & Service rate of hospital i \\ \hline
        \( N_i \) & Total capacity of hospital i \\ \hline
    \end{tabular}
    \caption{Fixed Parameters}
\end{table}

Having established the fixed parameters of the model, the hospitals' utilities need 
to be calculated. 
In order to do so a backwards induction approach will be used. 
The EMS aims to distribute the patients such that the mean waiting time of patients 
is minimal. 
This can be further interpreted as when the mean waiting time of hospital A equals 
the mean waiting time of hospital B. 
Thus, the minimal mean waiting time can be found for the values of \( p_A \) and 
\( p_B \) that solve the following equation:

\begin{equation}\label{Equal_Wait}
    W_q(\lambda_A, \mu_A, \hat{c_A}) = W_q(\lambda_B, \mu_B, \hat{c_B})
\end{equation}

Equation (\ref{Equal_Wait}) needs to be solved for all values of \( c_i \in \{1,2, 
\dots C_A\} \) and \( c_j \in \{1,2, \dots C_B\} \). 
Then, for every \( c_i \) and \( c_j \) the utility equation (\ref{Utilities}) has 
to be calculated for both hospitals. 
In order to solve it though, one must first estimate the probability 
\( P[(W_q)_{\{A, B\}}] \leq 4] \). 
That is the probability that the waiting time in the queue for one of the hospitals 
is less than 4 hours. 
For a multi-server system, the distribution of the waiting time can be given by 
equation \ref{Dist_Wait}. 
The above expression returns the probability that the waiting time in the queue 
is less than some time T.

\begin{equation}\label{Dist_Wait}
    P(W_q > T) = \frac{(\frac{\lambda}{\mu})^c P_0}{c!(1-\frac{\lambda}{c \mu})} 
    (e^{-(c \mu - \lambda)T})
\end{equation}

Consequently when incorporating equation (\ref{Dist_Wait}) into (\ref{Utilities}) 
a newer utility equation can be acquired:
 
\begin{equation}\label{Utilities2}
    U_{\hat{c_i}, \hat{c_j}} ^ {\{A, B\}} = 1 - \left| \left[ 
        \frac{(\frac{\lambda}{\mu})^c P_0}{c!(1-\frac{\lambda}{c \mu})} 
        \left( e^{-(c \mu - \lambda)T} \right) \right] - 0.05 \right|
\end{equation}

\begin{table}[h]
    \centering
    \begin{minipage}{.5\linewidth}
        A = 
        \begin{tabular}{|l|l|l|l|}
            \hline
            \( U_{1,1}^A \) & \( U_{1,2}^A \) & \dots & \( U_{1,C_2}^A \) \\ \hline
            \( U_{2,1}^A \) & \( U_{2,2}^A \) & \dots & \( U_{2,C_2}^A \) \\ \hline
            \vdots & \vdots & \( \ddots \) & \vdots \\ \hline
            \( U_{C_1,1}^A \) & \( U_{C_1,2}^A \) & \dots & \( U_{C_1,C_2}^A \) \\ \hline
        \end{tabular}
    \end{minipage}%
    \begin{minipage}{.5\linewidth}
        B = 
        \begin{tabular}{|l|l|l|l|}
            \hline
            \( U_{1,1}^B \) & \( U_{1,2}^B \) & \dots & \( U_{1,C_2}^B \) \\ \hline
            \( U_{2,1}^B \) & \( U_{2,2}^B \) & \dots & \( U_{2,C_2}^B \) \\ \hline
            \vdots & \vdots & \( \ddots \) & \vdots \\ \hline
            \( U_{C_1,1}^B \) & \( U_{C_1,2}^B \) & \dots & \( U_{C_1,C_2}^B \) \\ \hline
        \end{tabular}
    \end{minipage}
\end{table}  
