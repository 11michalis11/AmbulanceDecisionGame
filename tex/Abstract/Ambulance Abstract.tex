\documentclass{article}
\usepackage{soul}

\title{A game theoretic model of the behavioural gaming that takes place at the Emergency Medical System - Emergency Department interface}
\author{}
\date{}

\begin{document}
\maketitle

Emergency departments (EDs) in hospitals are usually under pressure to achieve a target amount of time that describes the arrival of patients and the time it takes to receive treatment. For example in the UK this is often set as 95\% of patients to be treated within 4 hours. There is empirical evidence to suggest that imposing targets in the ED results in gaming at the interface of care between the EMS and ED. If the ED is busy and a patient is stable in the ambulance, there is little incentive for the ED to accept the patient whereby the clock will start ticking on the 4 hour target. This in turn impacts on the ability of the EMS to respond to emergency calls.

This study explores the impact that this effect may have on an ambulance's utilisation and their ability to respond to emergency calls. More specifically multiple scenarios are examined where an ambulance service needs to distribute patients between neighbouring hospitals. The interaction between the hospitals and the ambulance service is defined in a game theoretic framework where the ambulance service has to decide how many patients to distribute to each hospital in order to minimise the occurrence of this effect. The methodology involves the use of a queueing model for each hospital that is used to inform the decision process of the ambulance service so as to create a game for which the Nash Equilibria can be calculated.

\end{document}